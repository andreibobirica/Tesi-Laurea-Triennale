% Load variables
\newcommand{\myUni}{Università degli Studi di Padova}
\newcommand{\myDepartment}{Dipartimento di Matematica ``Tullio Levi-Civita''}
\newcommand{\myFaculty}{Corso di Laurea in Informatica}
\newcommand{\myTitle}{Lorem ipsum dolor sit amet, consectetur adipisci elit.}
\newcommand{\myDegree}{Tesi di Laurea Triennale}
\newcommand{\profTitle}{Prof.}
\newcommand{\myProf}{Cognome Nome}
\newcommand{\graduateTitle}{Laureando}
\newcommand{\myName}{Bobirica Andrei Cristian}
\newcommand{\myStudentID}{1224449}
\newcommand{\myAA}{2023-2024}
\newcommand{\myLocation}{Padova}
\newcommand{\myTime}{Luglio 2024}
% Glossary
\newacronym{ddc}{DDC}{Data Driven Cooking}
\newglossaryentry{ddcg}{
    name={DDC},
    text={Data Driven Cooking},
    sort=ddc,
    description={Una piattaforma di cucina intelligente che utilizza i dati per ottimizzare e migliorare i processi di cottura. DDC fornisce funzionalità avanzate per i proprietari di forni, consentendo loro di monitorare e controllare i dispositivi in modo efficiente}
}

\newacronym{rtk}{RTK}{Redux Toolkit}
\newglossaryentry{rtkg}{
    name={RTK},
    text={Redux Toolkit},
    sort=rtk,
    description={Un toolkit per Redux che semplifica la scrittura della logica Redux e automatizza configurazioni complesse. RTK è stato utilizzato per gestire lo stato globale dell'applicazione in modo efficiente e strutturato}
}

\newacronym{cicd}{CI/CD}{continuous integration and continuous delivery}
\newglossaryentry{cicdg}{
    name={CI/CD},
    text={CI/CD},
    sort=cicd,
    description={Integrazione continua e consegna continua. L'integrazione continua (CI) è una pratica di sviluppo software in cui i membri del team integrano il loro lavoro frequentemente, con verifiche automatizzate per rilevare errori rapidamente. La consegna continua (CD) estende la CI, automatizzando ulteriormente il processo di rilascio per consentire la distribuzione di software in produzione in modo rapido e affidabile.}
}


\newglossaryentry{ddcserviceg}{
    name={DDC Service},
    sort=ddcservice,
    description={Un'applicazione destinata al personale tecnico e di manutenzione dei forni, offrendo strumenti avanzati per la gestione e il supporto dei dispositivi. DDC Service facilita la risoluzione dei problemi e l'ottimizzazione delle operazioni di servizio}
}

\newglossaryentry{servicedoveng}{
    name={Serviced Oven},
    sort=servicedoven,
    description={Nel contesto della applicazione DDC Service per Serviced Oven si intente un prodotto, in particolare un forno. Questo forno è un prodotto a cui un addetto Service presta servizzi di assistenza e riparazione. Un Utente Autenticato della app DDC Service ha tra i sui Serviced Ovens i prodotti a cui presta assistenza e servizi di Service. Un Serviced Oven è un prodotto fisico realmente esistente edentificato da un seriale.}
}

\newglossaryentry{productcodeg}{
    name={Product Code},
    sort=productcode,
    description={Un identificativo univoco assegnato a ogni prodotto per la tracciabilità e la gestione. Utilizzato nelle applicazioni per monitorare e gestire specifici modelli di forni}
}

\newglossaryentry{firmwareg}{
    name={Firmware},
    sort=firmware,
    description={Il software integrato nei dispositivi hardware, come i forni, che gestisce le loro operazioni fondamentali. Gli aggiornamenti del firmware migliorano le prestazioni e aggiungono nuove funzionalità ai dispositivi}
}

\newglossaryentry{navbarg}{
    name={Nav Bar},
    sort=navbar,
    description={La barra di navigazione dell'applicazione che consente agli utenti di accedere rapidamente alle diverse sezioni e funzionalità}
}

\newglossaryentry{tabbarg}{
    name={Tab Bar},
    sort=tabbar,
    description={Una barra di navigazione a schede che permette agli utenti di passare facilmente tra diverse schermate o funzionalità dell'applicazione}
}

\newglossaryentry{ecng}{
    name={ECN},
    sort=ecn,
    description={Per ECN si intende un codice che identifica il versionamente di uno specifico forno, un diverso versionamento potrebbe indicare caratteristiche diverse di un medesimo prodotto}
}

\newacronym{e2e}{E2E}{End To End}
\newglossaryentry{e2eg}{
    name={E2E},
    text={End To End},
    sort=e2e,
    description={Un tipo di test che verifica il funzionamento completo di un'applicazione dal punto di vista dell'utente finale, garantendo che tutte le componenti interagiscano correttamente}
}

\newglossaryentry{designsystemg}{
    name={Design System},
    sort=designsystem,
    description={Un insieme di linee guida, componenti riutilizzabili e strumenti per creare un'interfaccia utente coerente e unificata. Il Design System garantisce uniformità visiva e comportamentale in tutte le parti dell'applicazione}
}

\newglossaryentry{frontendg}{
    name={Frontend},
    sort=frontend,
    description={La parte dell'applicazione con cui gli utenti interagiscono direttamente. Comprende l'interfaccia utente e la logica di presentazione}
}

\newglossaryentry{backendg}{
    name={Backend},
    sort=backend,
    description={La parte dell'applicazione che gestisce la logica di business, il database, e le operazioni di server. Il backend supporta il frontend fornendo dati e funzionalità}
}

\newacronym{repo}{Repo}{Repository}
\newglossaryentry{repog}{
    name={Repo},
    text={Repository},
    sort=repo,
    description={Un archivio centralizzato dove il codice sorgente e altri file di un progetto vengono memorizzati e gestiti, spesso utilizzato con sistemi di controllo versione come Git}
}

\newacronym{rma}{RMA}{Return Merchandise Authorization}
\newglossaryentry{rmag}{
    name={RMA},
    text={Return Merchandise Authorization},
    sort=rma,
    description={Un processo che consente ai clienti di restituire prodotti difettosi o non conformi per la riparazione, la sostituzione o il rimborso}
}

\newacronym{devops}{DevOps}{Development and Operations}
\newglossaryentry{devopsg}{
    name={DevOps},
    text={Development and Operations},
    sort=devops,
    description={Una pratica che combina lo sviluppo software (Development) e le operazioni IT (Operations) per migliorare la collaborazione e la produttività, automatizzando i processi di sviluppo, test e distribuzione del software}
}

\newglossaryentry{designpatterng}{
    name={Design Pattern},
    sort=designpattern,
    description={Soluzioni riutilizzabili a problemi comuni di progettazione nel software. I design pattern facilitano la creazione di codice robusto e manutenibile}
}

\newglossaryentry{crplg}{
    name={cross-platform},
    sort=crossplatform,
    description={Un approccio allo sviluppo software che permette di creare applicazioni compatibili con più sistemi operativi o piattaforme con un unico codice sorgente}
}

\newglossaryentry{monorepog}{
    name={monorepo},
    sort=monorepo,
    description={Una strategia di gestione del codice sorgente in cui più progetti vengono memorizzati in un unico repository. Il monorepo facilita la condivisione del codice e la gestione delle dipendenze tra i progetti}
}

\newacronym{api}{API}{Application Programming Interface}
\newglossaryentry{apig}{
    name={API},
    text={Application Programming Interface},
    sort=api,
    description={Un insieme di procedure e strumenti per la creazione di software applicativo. Un'API consente a diverse applicazioni di comunicare tra loro, fornendo un'interfaccia per l'interazione con componenti software o hardware}
}

\newacronym{jwt}{JWT}{JSON Web Token}
\newglossaryentry{jwtg}{
    name={JWT},
    text={JSON Web Token},
    sort=jwt,
    description={Un formato compatto e sicuro per la trasmissione di informazioni tra parti come un oggetto JSON. I JWT sono spesso utilizzati per l'autenticazione e l'autorizzazione nelle applicazioni web}
}

\newacronym{sdk}{SDK}{Software Development Kit}
\newglossaryentry{sdkg}{
    name={SDK},
    text={Software Development Kit},
    sort=sdk,
    description={Un insieme di strumenti di sviluppo software in un unico pacchetto installabile, che facilita la creazione di applicazioni fornendo un compilatore, un debugger e a volte un framework software}
}

\newacronym{uml}{UML}{Unified Modeling Language}
\newglossaryentry{umlg}{
    name={UML},
    text={Unified Modeling Language},
    sort=uml,
    description={Un linguaggio di modellazione e specifica basato sul paradigma orientato agli oggetti, utilizzato per descrivere soluzioni analitiche e progettuali in modo conciso e comprensibile per una vasta gamma di destinatari}
}

\newglossaryentry{TermineSenzaAcronimo}{
    name={Nome del termine},
    sort=termine senza acronimo,
    description={Descrizione}
}


% Define custom colors
\definecolor{hyperColor}{HTML}{0B3EE3}
\definecolor{tableGray}{HTML}{F5F5F7}

% Set line height
\linespread{1.5}

% Custom hyphenation rules
\hyphenation {
    e-sem-pio
    ex-am-ple
}

% Images path
\graphicspath{{img/}}

% Force page color, as some editors set a grayish color as default
\pagecolor{white}

% Better spacing for footnotes
\setlength{\skip\footins}{5mm}
\setlength{\footnotesep}{5mm}

\setlength{\headheight}{14.5pt}
\addtolength{\topmargin}{-2.45pt}

% Add a subscript G to a glossary entry
\newcommand{\glox}{\textsubscript{\textbf{\textit{G }}}}

% If the subscript G is followed by a punctuation character, or anything else, you need to use \gloxspacing to prevent rendering issues, where the characters collide. Example in Chapter 7
\newcommand{\gloxspacing}{\hspace{-0.3em}}

% Improvements the paragraph command
\titleformat{\paragraph}
{\normalfont\normalsize\bfseries}{\theparagraph}{1em}{}
\titlespacing*{\paragraph}
{0pt}{3.25ex plus 1ex minus .2ex}{1.5ex plus .2ex}

% Define use case environment
\newcounter{usecasecounter} % define a counter
\setcounter{usecasecounter}{0} % set the counter to some initial value
% Parameters
% #1: ID
% #2: Nome
\newenvironment{usecase}[2]{
    \renewcommand{\theusecasecounter}{\usecasename #1}  % this is where the display of the counter is overwritten/modified
    \refstepcounter{usecasecounter} % increment counter
    \vspace{2em}
    \par \noindent % start new paragraph
    {\normalsize \textbf{\usecasename #1: #2}} % display the title before the content of the environment is displayed
    \vspace{.5em}
}{
    \medskip
}
\newcommand{\usecasename}{UC}
\newcommand{\usecaseactors}[1]{\textbf{\\Attori Principali:} #1}
\newcommand{\usecasepre}[1]{\textbf{\\Precondizioni:} #1}
\newcommand{\usecasedesc}[1]{\textbf{\\Descrizione:} #1}
\newcommand{\usecasepost}[1]{\textbf{\\Postcondizioni:} #1}
\newcommand{\usecasealt}[1]{\textbf{\\Scenario Alternativo:} #1}

% Define risks environment
\newcounter{riskcounter} % define a counter
\setcounter{riskcounter}{0} % set the counter to some initial value
% Parameters
% #1: Title
\newenvironment{risk}[1]{
    \refstepcounter{riskcounter} % increment counter
    \par \noindent % start new paragraph
    \textbf{\arabic{riskcounter}. #1} % display the title before the content of the environment is displayed
}{
    \par\medskip
}
\newcommand{\riskname}{Rischio}
\newcommand{\riskdescription}[1]{\textbf{\\Descrizione:} #1.}
\newcommand{\risksolution}[1]{\textbf{\\Soluzione:} #1.}

% Apply fancy styling to pages
\pagestyle{fancy}
\fancyhf{}
\fancyhead[L]{\leftmark} % Places Chapter N. Chatper title on the top left
\fancyfoot[C]{\thepage} % Page number in the center of the footer

% Adds a blank page while increasing the page number
\newcommand\blankpage{ 
\clearpage
    \begingroup
    \null
    \thispagestyle{empty}
    \hypersetup{pageanchor=false}
    \clearpage
\endgroup
}

% Increase page numbering
\newcommand\increasepagenumbering{
    \addtocounter{page}{+1}
}

% Make glossaries and bibliography
\makeglossaries
\bibliography{references/bibliography}
\defbibheading{bibliography} {
    \cleardoublepage
    \phantomsection
    \addcontentsline{toc}{chapter}{\bibname}
    \chapter*{\bibname\markboth{\bibname}{\bibname}}
}

% Code blocks handling w/ table of codes
\makeatletter
\ifdefined\NR@chapter
  \expandafter\@firstoftwo
\else
  \expandafter\@secondoftwo
\fi{\patchcmd\NR@chapter}{\patchcmd\@chapter}
  {\addtocontents{lot}{\protect\addvspace{10\p@}}}
  {\addtocontents{lot}{\protect\addvspace{10\p@}}%
   \addtocontents{lol}{\protect\addvspace{10\p@}}}
  {}{}
\makeatother

\renewcommand\listingscaption{Codice}
\renewcommand\listoflistingscaption{Elenco dei codici sorgenti}
\counterwithin*{listing}{chapter}
\renewcommand\thelisting{\thechapter.\arabic{listing}}

% Set up hyperlinks
\hypersetup{
    colorlinks=true,
    linktocpage=true,
    pdfstartpage=1,
    pdfstartview=,
    breaklinks=true,
    pdfpagemode=UseNone,
    pageanchor=true,
    pdfpagemode=UseOutlines,
    plainpages=false,
    bookmarksnumbered,
    bookmarksopen=true,
    bookmarksopenlevel=1,
    hypertexnames=true,
    pdfhighlight=/O,
    allcolors = hyperColor
}

% Set up captions
\captionsetup{
    tableposition=top,
    figureposition=bottom,
    font=small,
    format=hang,
    labelfont=bf
}