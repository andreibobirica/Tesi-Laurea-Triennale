\chapter{Tecnologie utilizzate}
\label{chap:tecnologie_utilizzate}
Questo capitolo esplora le tecnologie chiave adottate nel contesto dello sviluppo dell'applicazione \gls{ddcserviceg}\glox.
Vengono presentati i linguaggi di programmazione, i \textit{framework}, gli strumenti di sviluppo, le piattaforme di collaborazione e gestione, oltre alle soluzioni \textit{cloud} e \gls{devopsg}\glox impiegate per supportare e ottimizzare il processo di sviluppo.
Ogni tecnologia è discussa nel contesto del suo ruolo nell'ecosistema di sviluppo dell'applicazione, evidenziando come contribuisca alla scalabilità, alla manutenibilità e alla coerenza del codice, nonché al miglioramento complessivo dell'efficienza operativa dell'azienda.
\section{Linguaggi di programmazione}
\begin{itemize}
    \item \textbf{TypeScript}: è un linguaggio di programmazione \textit{open-source} sviluppato da Microsoft.
    TypeScript è un \textit{superset} di \textit{JavaScript} che aggiunge la tipizzazione statica opzionale e altre funzionalità moderne, rendendo il codice più robusto e manutenibile. 
    Questo linguaggio è stato utilizzato per lo sviluppo dell'applicazione \textit{DDC Service}, sia per la parte \gls{frontendg}\glox che per l'integrazione con i servizi \gls{backendg}\glox, garantendo una maggiore affidabilità e scalabilità del codice.
\end{itemize}

\section{\textit{Framework} in uso}
\begin{itemize}
    \item \textbf{Expo}: è un \textit{framework open-source} per la creazione di applicazioni \textit{React Native}.
    Facilita lo sviluppo di applicazioni \textit{mobile} fornendo strumenti e librerie preconfigurate.
    \textit{Expo} è stato utilizzato per lo sviluppo delle applicazioni \textit{Android} e \textit{IOS}, permettendo di scrivere il codice una sola volta e distribuirlo su entrambe le piattaforme in modo efficiente.
    \item \textbf{Next.js}: un \textit{framework} di sviluppo \textit{React} per la creazione di applicazioni \textit{Web}.
    Supporta il \textit{rendering} lato \textit{server} e la generazione di siti statici, migliorando così le prestazioni e l'ottimizzazione per i motori di ricerca (\textit{SEO}).
    \item \textbf{React}: è una libreria \textit{JavaScript} per la costruzione di interfacce utente sviluppata da \textit{Facebook}. React si distingue per la sua architettura basata su componenti e l'uso del \textit{virtual DOM}, che rendono lo sviluppo di interfacce utente reattive ed efficienti. In particolare, React è stato utilizzato come base per le applicazioni, consentendo la creazione di componenti riutilizzabili che migliorano la coerenza e la manutenibilità del codice.
    \item \textbf{React Native}: è un framework open-source per lo sviluppo di applicazioni mobili creato da \textit{Facebook}. React Native permette di utilizzare React e \textit{TypeScript} per costruire applicazioni native per \textit{IOS} e \textit{Android}. Unox utilizza React Native per sviluppare applicazioni mobili, permettendo al \textit{team} di scrivere il codice una sola volta e distribuirlo su entrambe le piattaforme, semplificando e ottimizzando gli sforzi di sviluppo.
    \item \textbf{NodeJS}: è una piattaforma di \textit{runtime open-source} basata su \textit{JavaScript V8} di \textit{Chrome}, progettata per costruire applicazioni di rete veloci e scalabili. Unox utilizza NodeJS per l'esecuzione del codice \textit{TypeScript} e come gestore di pacchetti. Facilita le operazioni lato \textit{server}, consentendo una gestione efficiente di compiti come il \textit{rendering} del server e lo sviluppo di \gls{api}\glox, migliorando le prestazioni e la scalabilità dell'applicazione.
\end{itemize}

\section{Tecnologie per \textit{monorepo}}
\begin{itemize}
\item \textbf{NPM Workspaces}: una funzionalità di NPM che consente di gestire più pacchetti all'interno di un unico \gls{repog}\glox.
\\\textit{NPM Workspaces} è stato utilizzato per organizzare i vari pacchetti del progetto \gls{ddcserviceg}\glox, semplificando la gestione delle dipendenze e migliorando l'efficienza dello sviluppo.
\item \textbf{NX}: un set di strumenti per la gestione di \gls{monorepog}\glox che facilita lo sviluppo, il test e la manutenzione di applicazioni e librerie su larga scala.
Nel progetto \textit{DDC Service}, NX è stato implementato nella parte finale per la gestione del \textit{monorepo}, per organizzare e gestire le dipendenze del codice e migliorando la modularità e la coerenza del progetto.
\end{itemize}

\section{Librerie utilizzate}
\begin{itemize}
    \item \textbf{Solito}: una libreria che permette di condividere il codice tra applicazioni \textit{Next} ed \textit{Expo}, riducendo la duplicazione del codice e semplificando la manutenzione.
    Nel contesto del progetto \gls{ddcserviceg}\glox, \textit{Solito} è stato impiegato per ottimizzare la condivisione del codice tra le piattaforme \textit{web e mobile}, implementando una gestione del \textit{routing} comune tra le pagine.
    Ciò ha permesso di mantenere una struttura di navigazione coerente e una logica di gestione dei percorsi uniforme, migliorando l'esperienza dell'utente e semplificando lo sviluppo e la manutenzione dell'applicazione su entrambe le piattaforme.\footcite{site:solito}
    \item \textbf{Moti}: una libreria di animazioni per \textit{React Native} che facilita la creazione di animazioni complesse e fluide.
    È stata utilizzata nel progetto \textit{DDC Service} per migliorare l'interazione dell'utente e l'aspetto visivo delle applicazioni mobili.\footcite{site:moti}
    \item \textbf{Dripsy}: una libreria per la gestione dello stile di \textit{UI} per \textit{React Native} e \textit{Web}.
    Dripsy permette di definire uno stile una sola volta e applicarlo ovunque, supportando la creazione di interfacce responsive che si adattano automaticamente a diverse dimensioni di schermo. È compatibile con Expo, Vanilla React Native e Next.js, offrendo un supporto completo per TypeScript e facilitando l'implementazione di temi personalizzati e varianti di tema. Con una semplice \gls{api}\glox, è possibile definire stili tematici e responsivi in una sola riga di codice. Supporta anche modalità scura e personalizzazione dei colori.\footcite{site:dripsy}
    \item \textbf{Redux}: una libreria per la gestione dello stato delle applicazioni \textit{JavaScript}.
    \textit{Redux} è utilizzato per mantenere uno stato globale consistente e prevedibile nell'applicazione, facilitando la gestione dello stato complesso e la sincronizzazione dei dati tra i vari componenti.
    \item \textbf{Redux Toolkit (RTK)}: una serie di strumenti e convenzioni per semplificare l'uso di \textit{Redux}. \textit{RTK} include funzioni per la creazione di \textit{slice} di stato, \textit{middleware} personalizzati e la gestione di operazioni asincrone, rendendo lo sviluppo con \textit{Redux} più efficiente e meno soggetto a errori.
    \item \textbf{GraphQL}: un linguaggio di query per \textit{API}\glox.
    \textit{GraphQL} è utilizzato per ottenere dati in modo efficiente e flessibile, consentendo di specificare esattamente quali dati sono necessari.
    Nel progetto \gls{ddcserviceg}\glox, \textit{GraphQL} facilita la comunicazione tra il \gls{frontendg}\glox e il \gls{backendg}\glox, migliorando le performance e riducendo la quantità di dati trasferiti.
    \item \textbf{GraphQL Code Generator}: uno strumento per generare tipi \textit{TypeScript} per le query, le mutazioni e i frammenti definiti nello schema \textit{GraphQL}. Questo migliora la sicurezza del tipo e riduce gli errori di \textit{runtime}, mantenendo il codice sincronizzato con lo schema \textit{GraphQL}.
\end{itemize}


\section{Strumenti di sviluppo}
\begin{itemize}
\item \textbf{Visual Studio Code}: un \textit{editor} di codice sorgente sviluppato da \textit{Microsoft}, altamente estensibile e utilizzato per una varietà di linguaggi di programmazione.
\item \textbf{Xcode}: un ambiente di sviluppo integrato (\textit{IDE}) di \textit{Apple} per \textit{macOS}, utilizzato per sviluppare software per \textit{IOS}, \textit{macOS}, \textit{watchOS} e \textit{tvOS}.
\item \textbf{Android Studio}: un \textit{IDE} ufficiale per lo sviluppo di applicazioni \textit{Android}, fornito da \textit{Google}. Viene utilizzato per scrivere, eseguire il debug e testare le applicazioni \textit{Android}.
\item \textbf{Prettier}: uno strumento di formattazione del codice che aiuta a mantenere uno stile di codice coerente in tutti i progetti.
\item \textbf{Cocoapods}: un gestore di dipendenze per \textit{Swift} e \textit{Objective-C Cocoa projects}. Viene utilizzato per integrare librerie di terze parti nei progetti \textit{IOS}.
\end{itemize}

\section{Piattaforme di collaborazione e gestione}
\begin{itemize}
\item \textbf{Git}: viene utilizzato come sistema di controllo delle versioni per il tracciamento delle modifiche al codice sorgente.
\item \textbf{Microsoft Teams}: adottato come strumento di comunicazione e collaborazione in tempo reale all'interno dell'azienda, facilitando le discussioni, le videochiamate e la condivisione di documenti. Viene utilizzato anche per la calendarizzazione di eventi e meeting.
\end{itemize}

\section{Piattaforme \textit{cloud} e \textit{DevOps}}
\begin{itemize}
\item \textbf{Microsoft Azure}: una piattaforma cloud utilizzata per l'hosting di applicazioni, servizi e dati aziendali. Viene utilizzato da Unox per l'hosting di alcuni dei servizi principali. Dalla suite di Azure, viene utilizzato anche \textit{Azure DevOps} per la gestione delle attività di sviluppo software, tra cui la gestione dei repository Git, delle build e delle attività.
\item \textbf{AWS}: \textit{Amazon Web Services} (\textit{AWS}) è un altro servizio cloud utilizzato per le risorse di calcolo, archiviazione e servizi di rete. Alcuni dei servizi secondari di Unox sono ospitati su AWS.
\item \textbf{Amplify}: è una piattaforma di sviluppo di applicazioni cloud che facilita l'integrazione di funzionalità come autenticazione, \textit{API}, \textit{storage} e altro ancora.
In questo progetto, \textit{Amplify} è stato utilizzato per scaricare automaticamente le chiavi di accesso, migliorando la sicurezza e semplificando la gestione delle credenziali.
Inoltre, Amplify è stato impiegato per implementare le notifiche \textit{push}, permettendo una comunicazione efficace e tempestiva con gli utenti dell'applicazione.
\end{itemize}

\section{Strumenti di \textit{design}}
\begin{itemize}
\item \textbf{Figma}: uno strumento di design collaborativo utilizzato per la progettazione delle interfacce utente. Facilita la collaborazione tra \textit{designer} e sviluppatori e permette di creare e condividere facilmente prototipi e \textit{design}.
\end{itemize}


\newpage