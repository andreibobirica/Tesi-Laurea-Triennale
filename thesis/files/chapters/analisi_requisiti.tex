\chapter{Analisi dei requisiti}
\label{chap:analisi_requisiti}

\subsubsection{Caratteristiche degli utenti}
La user base attesa è un subset degli utenti del sito \href{https://unox.com}{unox.com}. Pertanto sarà un'utenza con interesse per il mondo della cucina di alto livello: pasticceria, gastronomia, cucine ad alti volumi. In particolare, ci si aspetta che gli utenti che desiderano calcolare i consumi dei prodotti prestino una particolare attenzione all'efficienza enegetica e/o al possibile risparmio economico.

\subsubsection{Vincoli generali}
L'utente, per usufruire del servizio, necessita di un browser web e di una connessione ad internet.

Il prodotto finale sarà  quindi un componente \textit{React} per guidare gli utenti nel processo di selezione del proprio prodotto e nell'inserimento dei dati relativi al consumo. Pensando a come questo sarebbe stato effettivamente utilizzato, ho individuato i seguenti attori:
\begin{description}
	\item[Utente generico] L'utente generico è un qualsiasi utilizzatore del sito web di Unox che accede al servizio.
	\item[Cliente] Il Cliente è un utente generico. Questa è una distinzione puramente di Business e non garantisce accesso a nessuna funzionalità diversa dall'utente generico, allo stato attuale della specifica.
	\item[Commerciale]  Il Commerciale è un addetto alle vendite di Unox. Anche questo attore esiste solo per differenziare a livello Business e non ha nessun impatto a livello di sistema.
	\item[Fornitore di informazioni sull'energia] Un servizio esterno in grado di fornire prezzi di gas e corrente elettrica. È un attore secondario.
\end{description}
Durante le riunioni ho individuato i seguenti use case per la web-app da sviluppare:
\paragraph*{UC1}

\section{Casi d'uso}
Per lo studio dei casi di utilizzo del prodotto sono stati creati dei diagrammi.
I diagrammi dei casi d'uso (in inglese \textit{Use Case Diagram}) sono diagrammi di tipo \gls{uml} dedicati alla descrizione delle funzioni o servizi offerti da un sistema, così come sono percepiti e utilizzati dagli attori che interagiscono col sistema stesso.

Per comprendere le interazioni dell'utente con il prodotto, sono stati sviluppati diagrammi dei casi d'uso. 
Questi diagrammi, noti anche come \textit{Use Case Diagram} nell'ambito \gls{uml}, sono strumenti essenziali per descrivere le funzionalità o i servizi offerti dal sistema, così come percepiti e utilizzati dagli attori che interagiscono con esso.
\begin{figure}[H]
    \vspace{2em}
    \centering
    \includegraphics[alt={Testo alternativo dell'immagine}, width=0.75\columnwidth]{img/usecase/scenario-principale.jpeg}
    \caption{Use Case 0: Scenario principale}
    \label{fig:scenario_principale}
\end{figure}

\begin{usecase}{0}{Scenario principale}
    \usecaseactors{Sviluppatore applicativi.}
    \usecasepre{Lo sviluppatore è entrato nel plugin di simulazione all'interno dell'IDE.}
    \usecasedesc{La finestra di simulazione mette a disposizione i comandi per configurare, registrare o eseguire un test.}
    \usecasepost{Il sistema è pronto per permettere una nuova interazione.}
    \label{uc:uc_scenario_principale}
\end{usecase}

\begin{usecase}{1}{Gestione Utente}
    \usecaseactors{Amministratore, Utente Registrato.}
    \usecasepre{L'utente deve essere autenticato nel sistema.}
    \usecasedesc{L'utente può gestire le informazioni del proprio profilo.}
    \usecasepost{Le modifiche vengono salvate nel sistema.}
    \usecasealt{Se l'utente non è autenticato, visualizza un messaggio di errore.}
    \label{uc:uc_casi_uso}
\end{usecase}

\begin{usecase}{2}{Creazione Prodotto}
    \usecaseactors{Amministratore.}
    \usecasepre{L'amministratore ha effettuato l'accesso al sistema.}
    \usecasedesc{L'amministratore può aggiungere un nuovo prodotto al catalogo.}
    \usecasepost{Il nuovo prodotto viene aggiunto con successo.}
    \usecasealt{Se i campi obbligatori non sono compilati, visualizza un messaggio di errore.}
    \label{uc:uc_creazione_prodotto}
\end{usecase}

\section{Tracciamento dei requisiti}
Da un'attenta analisi dei requisiti e degli use case effettuata sul progetto è stata stilata la tabella che traccia i requisiti in rapporto agli use case.\\
Sono stati individuati diversi tipi di requisiti e si è quindi fatto utilizzo di un codice identificativo per distinguerli.\\
Il codice dei requisiti, dove ogni requisito è identificato con il carattere \textbf{R}, è così strutturato:
\begin{enumerate}
    \item[\textbf{F}:] Funzionale.
    \item[\textbf{Q}:] Qualitativo.
    \item[\textbf{V}:] Di vincolo.
    \item[\textbf{N}:] Obbligatorio (necessario).
    \item[\textbf{D}:] Desiderabile.
    \item[\textbf{Z}:] Opzionale.
\end{enumerate}

Nelle tabelle \ref{tab:requisiti_funzionali}, \ref{tab:requisiti_qualitativi} e \ref{tab:requisiti_vincolo} sono riassunti i requisiti e il loro tracciamento con gli use case delineati in fase di analisi.

\section{Tabelle dei requisiti}
\begin{center}
    \rowcolors{1}{}{tableGray}
    \begin{longtable}{|p{2.25cm}|p{7.75cm}|p{2.25cm}|}
    \hline
    %\rowcolor{hyperColor!5}
    \multicolumn{1}{|c|}{\textbf{Requisito}} & \multicolumn{1}{c|}{\textbf{Descrizione}} & \multicolumn{1}{c|}{\textbf{Use Case}}\\
    \hline 
    \endfirsthead
    \rowcolor{white}
    \multicolumn{3}{c}{{\bfseries \tablename\ \thetable{} -- Continuo della tabella}}\\
    \hline
    %\rowcolor{hyperColor!5}
    \multicolumn{1}{|c|}{\textbf{Requisito}} & \multicolumn{1}{c|}{\textbf{Descrizione}} & \multicolumn{1}{c|}{\textbf{Use Case}}\\
    \hline 
    \endhead
    \hline
    \rowcolor{white}
    \multicolumn{3}{|r|}{{Continua nella prossima pagina...}}\\
    \hline
    \endfoot
    \endlastfoot
    
    RFN-1 & L’interfaccia permette di configurare il tipo di sonde del test & UC1 \\
    \hline
    \hiderowcolors
    \caption{Tabella del tracciamento dei requisiti funzionali.}
    \label{tab:requisiti_funzionali}
    \end{longtable}
\end{center}

\begin{center}
    \rowcolors{1}{}{tableGray}
    \begin{longtable}{|p{2.25cm}|p{7.75cm}|p{2.25cm}|}
    \hline
    %\rowcolor{hyperColor!5}
    \multicolumn{1}{|c|}{\textbf{Requisito}} & \multicolumn{1}{c|}{\textbf{Descrizione}} & \multicolumn{1}{c|}{\textbf{Use Case}}\\
    \hline 
    \endfirsthead
    \rowcolor{white}
    \multicolumn{3}{c}{{\bfseries \tablename\ \thetable{} -- Continuo della tabella}}\\
    \hline
    %\rowcolor{hyperColor!5}
    \multicolumn{1}{|c|}{\textbf{Requisito}} & \multicolumn{1}{c|}{\textbf{Descrizione}} & \multicolumn{1}{c|}{\textbf{Use Case}}\\
    \hline 
    \endhead
    \hline
    \rowcolor{white}
    \multicolumn{3}{|r|}{{Continua nella prossima pagina...}}\\
    \hline
    %\caption{Tabella del tracciamento dei requisiti qualitativi.}
    \endfoot
    \endlastfoot
    
    RQD-1n & Le prestazioni del simulatore hardware deve garantire la giusta esecuzione dei test e non la generazione di falsi negativi & - \\
    \hline
    RQD-1n & Le prestazioni del simulatore hardware deve garantire la giusta esecuzione dei test e non la generazione di falsi negativi & - \\
    \hline
    RQD-1n & Le prestazioni del simulatore hardware deve garantire la giusta esecuzione dei test e non la generazione di falsi negativi & - \\
    \hline
    RQD-1n & Le prestazioni del simulatore hardware deve garantire la giusta esecuzione dei test e non la generazione di falsi negativi & - \\
    \hline
    RQD-1n & Le prestazioni del simulatore hardware deve garantire la giusta esecuzione dei test e non la generazione di falsi negativi & - \\
    \hline
    RQD-1n & Le prestazioni del simulatore hardware deve garantire la giusta esecuzione dei test e non la generazione di falsi negativi & - \\
    \hline
    \hiderowcolors
    \caption{Tabella del tracciamento dei requisiti qualitativi.}
    \label{tab:requisiti_qualitativi}
    \end{longtable}
\end{center}

\begin{center}
    \rowcolors{1}{}{tableGray}
    \begin{longtable}{|p{2.25cm}|p{7.75cm}|p{2.25cm}|}
    \hline
    %\rowcolor{hyperColor!5}
    \multicolumn{1}{|c|}{\textbf{Requisito}} & \multicolumn{1}{c|}{\textbf{Descrizione}} & \multicolumn{1}{c|}{\textbf{Use Case}}\\
    \hline 
    \endfirsthead
    \rowcolor{white}
    \multicolumn{3}{c}{{\bfseries \tablename\ \thetable{} -- Continuo della tabella}}\\
    \hline
    %\rowcolor{hyperColor!5}
    \multicolumn{1}{|c|}{\textbf{Requisito}} & \multicolumn{1}{c|}{\textbf{Descrizione}} & \multicolumn{1}{c|}{\textbf{Use Case}}\\
    \hline 
    \endhead
    \hline
    \rowcolor{white}
    \multicolumn{3}{|r|}{{Continua nella prossima pagina...}}\\
    \hline
    \endfoot
    \endlastfoot
    
    RVO-1 & La libreria per l'esecuzione dei test automatici deve essere riutilizzabile & - \\
    \hline
    \hiderowcolors
    \caption{Tabella del tracciamento dei requisiti di vincolo.}
    \label{tab:requisiti_vincolo}
    \end{longtable}
\end{center}

\newpage


