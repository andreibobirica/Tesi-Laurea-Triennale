\chapter{Analisi dei requisiti}
\label{chap:analisi_requisiti}

\section{Caratteristiche degli utenti}
Per esaminare la \textit{user base} dell'app \gls{ddcserviceg}\glox bisogna tenere in considerazione 
che tutti i dati di cui l'applicazione fa uso saranno reperibili dal \gls{backendg}\glox associato all'applicazione.
Le basi di dati con cui il \textit{backend} andrà a comunicare saranno popolate da piattaforme esterne ignote alla app in questione.
Il sistema di autenticazione non dovrà tenere in considerazione di tipologie di utenti speciali come \textit{Admin} o \textit{SysAdmin}.

Questi utenti possono far parte del personale Unox oppure possono essere professionisti con cui Unox collabora.
Sono persone che richiedono di accedere alla app sia da \textit{Desktop} tramite \textit{web-app} mentre sono in ufficio, che da \textit{app} sul proprio \textit{smartphone} mentre sono in mobilità.
La user base della app è composta da personale tecnico, personale responsabile alla vendita e utenti addetti al \textit{Service}.
Non esiste però la necessità lato autenticazione di rendere il processo di registrazione o di \textit{login} diverso in base alla funzione del utente o in base alla sua appartenenza.
In linea generica tutti gli utenti registrati hanno gli stessi permessi.
In futuro l'applicazione necessiterà di un sistema per rendere un utente registrato verificato o meno, però per questo progetto questa richiesta non è una necessità.

\section{Vincoli generali}
Date le premesse sulle caratteristiche degli utenti della applicazione, definiremo quindi solo due categorie di utenti: Utente Autenticato, Utente Non Autenticato.
Ho individuato i seguenti attori:
\begin{description}
	\item[Utente Non Autenticato] L'Utente Non Autenticato è una tipologia di utente che o non si è ancora registrato e quindi non possiede delle credenziali oppure possiede delle credenziali ma deve ancora far il processo di \textit{login} detto \textit{signin}.
	\item[Utente Autenticato] L'Utente Autenticato è una tipologia di utente che possiede delle credenziali valide e che ha completato con successo il processo di \textit{login} detto \textit{signin}.
\end{description}

\section{Casi d'uso}
I seguenti casi d'uso sono state formulati solo per le funzionalità che sono state realmente realizzate in quanto le funzionalità opzionali per le quali non si ha avuto tempo non sono state tenute in considerazione per l'analisi dei requisiti.

\begin{usecase}{0}{Autenticazione}
    \usecaseactors{Utente Non Autenticato}
    \usecasepre{Un Utente Non Autenticato vuole accedere alle funzionalità della \textit{app}}
    \usecasedesc{Questo scenario descrive un utente che vuole accedere alla funzionalità della \textit{app} ma che non ne ha i permessi, per tanto deve fare la registrazione (\textit{singup}) o il \textit{login} (\textit{singin})}
    \usecasepost{l'utente accede al processo di registrazione o di \textit{login}}
    \label{uc:uc_scenario_autenticazione}
\end{usecase}

\begin{usecase}{0.1}{Registrazione}
    \usecaseactors{Utente Non Autenticato}
    \usecasepre{Un Utente Non Autenticato NON possiede delle credenziali per poter accedere alla \textit{app}.}
    \usecasedesc{Questo scenario descrive un utente che non possiede credenziali di autenticazione e che non riesce ad accedere alle funzionalità della \textit{app}, per tanto intraprende il processo di registrazione con cui alla fine riuscirà ad avere delle credenziali valide.}
    \usecasepost{L'utente ora ha delle credenziali valide con cui poter effettuare il \textit{login} e diventare un Utente Autenticato}
    \usecasealt{Se l'utente non fornisce con correttezza tutti i dati necessari alla registrazione, visualizza un messaggio di errore.}
    \label{uc:uc_scenario_autenticazione_reg}
\end{usecase}

\begin{usecase}{0.2}{Login}
    \usecaseactors{Utente Non Autenticato}
    \usecasepre{Un Utente non Autenticato che possiede delle credenziali valide per l'autenticazione.}
    \usecasedesc{Tramite questo scenario l'utente fornisce le sue credenziali per l'autenticazione e completa il processo di \textit{login}.}
    \usecasepost{L'utente è diventato un Utente Autenticato}
    \usecasealt{Se l'utente fornisce delle credenziali sbagliate, visualizza un messaggio di errore.}
    \label{uc:uc_scenario_autenticazione_login}
\end{usecase}

\begin{usecase}{0.3}{Recover Password}
    \usecaseactors{Utente Non Autenticato}
    \usecasepre{Un Utente non Autenticato che possiede delle credenziali NON valide per l'autenticazione.}
    \usecasedesc{Tramite questo scenario l'utente fornisce le sue credenziali in maniera parziale e tramite un processo riesce a ripristinarle per poter avere delle credenziali valide.}
    \usecasepost{L'utente ha delle credenziali valide.}
    \label{uc:uc_scenario_autenticazione_recover}
\end{usecase}
\pagebreak
\begin{usecase}{1}{Visualizzazione Prodotto}
    \usecaseactors{Utente Autenticato}
    \usecasepre{Un utente ha completato con successo il processo di autenticazione.}
    \usecasedesc{L'utente visualizza le informazioni di un determinato prodotto incluse le sue parti commerciali e i dettagli tecnici a lui associati.}
    \usecasepost{L'utente ha visualizzato le informazioni collegate al prodotto richiesto.}
    \usecasealt{Se il prodotto richiesto non è esistente, viene visualizzato un messaggio di errore.}
    \label{uc:uc_scenario_prodotto}
\end{usecase}

\begin{usecase}{2}{Visualizzazione Serviced Oven}
    \usecaseactors{Utente Autenticato}
    \usecasepre{Un utente ha completato con successo il processo di autenticazione e ha tra i suoi forni alcuni dispositivi categorizzati come  \textit{Serviced Oven}, cioè forni a cui l'utente presta assistenza o manutenzione.}
    \usecasedesc{L'utente visualizza le informazioni di un determinato \gls{servicedoveng}\glox di cui ne ha i permessi.}
    \usecasepost{L'utente ha visualizzato le informazioni collegate al \textit{Serviced Oven} richiesto.}
    \usecasealt{Se il prodotto richiesto non è esistente, viene visualizzato un messaggio di errore.}
    \label{uc:uc_scenario_servicedoven}
\end{usecase}

\pagebreak
\section{Tracciamento dei requisiti}
Da un'attenta analisi dei requisiti e degli use case effettuata sul progetto è stata stilata la tabella che traccia i requisiti in rapporto agli use case.\\
Sono stati individuati diversi tipi di requisiti e si è quindi fatto utilizzo di un codice identificativo per distinguerli.\\
Il codice dei requisiti, dove ogni requisito è identificato con il carattere \textbf{R}, è così strutturato:
\begin{enumerate}
    \item[\textbf{F}:] Funzionale.
    \item[\textbf{Q}:] Qualitativo.
    \item[\textbf{V}:] Di vincolo.
    \item[\textbf{N}:] Obbligatorio (necessario).
    \item[\textbf{D}:] Desiderabile.
\end{enumerate}

Nelle tabelle \ref{tab:requisiti_funzionali}, \ref{tab:requisiti_qualitativi} e \ref{tab:requisiti_vincolo} sono riassunti i requisiti e il loro tracciamento con gli use case delineati in fase di analisi.

\section{Tabelle dei requisiti}


\begin{center}
    \rowcolors{1}{}{tableGray}
    \begin{longtable}{|p{2.25cm}|p{7.75cm}|p{2.25cm}|}
    \hline
    %\rowcolor{hyperColor!5}
    \multicolumn{1}{|c|}{\textbf{Requisito}} & \multicolumn{1}{c|}{\textbf{Descrizione}} & \multicolumn{1}{c|}{\textbf{Use Case}}\\
    \hline 
    \endfirsthead
    \rowcolor{white}
    \multicolumn{3}{c}{{\bfseries \tablename\ \thetable{} -- Continuo della tabella}}\\
    \hline
    %\rowcolor{hyperColor!5}
    \multicolumn{1}{|c|}{\textbf{Requisito}} & \multicolumn{1}{c|}{\textbf{Descrizione}} & \multicolumn{1}{c|}{\textbf{Use Case}}\\
    \hline 
    \endhead
    \hline
    \rowcolor{white}
    \multicolumn{3}{|r|}{{Continua nella prossima pagina...}}\\
    \hline
    %\caption{Tabella del tracciamento dei requisiti qualitativi.}
    \endfoot
    \endlastfoot
    RFN-1 & La funzionalità di autenticazione deve essere accessibile solo a Utenti Non Autenticati & UC0 \\
    \hline
    RFN-2 & L'utente deve essere in grado di accedere alla pagina registrazione detta \textit{signup}. Questa pagina deve dare la possibilità di inserire i campi \textit{Name, Surname, Email, Password, Confirm Password, Company Name, Business Type, Address, Phone Number, Contry, Language} e scelte \textit{checkbox} GDPR e consenso Marketing & UC0.1 \\
    \hline
    RFN-2.1 & Nel \textit{form} per la registrazione devono esserci filtri per il controllo dei dati immessi come \textit{input} & UC0.1 \\
    \hline
    RFN-2.2 & Una volta immessi i dati nel \textit{form} della pagina di registrazione questa deve mostrare un messaggio che indica al utente di attivare l'account con la conferma della \textit{email} & UC0.1 \\
    \hline
    RFD-2.2.1 & Una volta confermata la email il processo di registrazione è completato, è richiesto che la app faccia automaticamente il \textit{login} con l'account appena creato, quindi ad ogni registrazione avviene il \textit{login} automatico & UC0.1 \\
    \hline
    RFN-3 & Nella pagina di \textit{login} l'utente presenta le sue credenziali nel apposito \textit{form}, cioè \textit{email e password} per poter diventare un Utente Autenticato & UC0.2 \\
    \hline
    RFN-3.1 & Nella pagina di \textit{login} nel caso in cui l'utente abbia inserito le credenziali non valide deve apparire un messaggio di errore & UC0.2 \\
    \hline
    RFN-4 & Deve esistere una pagina collegata alla pagina di \textit{login} che permetta di fare il processo di ripristino \textit{password}, detto \textit{Recover Password}, con questo processo l'utente inserisce la propria \textit{email} e dopo un processo di verifica gli viene confermato il ripristino della \textit{password} & UC0.3 \\
    \hline
    RFN-5 & La funzionalità di Prodotti definisce la \textit{feature} di visualizzazione di prodotti di Unox, sono tutti i prodotti esistenti e sono oggetti astratti da catalogo, non macchine reali, è richiesto solo la visualizzazione di un singolo prodotto nella sua \textit{Product Page} & UC1 \\
    \hline
    RFN-5.1 & Un prodotto deve essere identificato dal proprio \gls{productcodeg}\glox, se lo si identifica solo con il \textit{product code} si intende che si vuole visualizzare il prodotto con l'ultimo \gls{ecng}\glox & UC1 \\
    \hline
    RFN-5.2 & Nel caso in cui si identifichi un prodotto tramite il suo seriale, si intende che si vuole una istanza di un prodotto, quindi un prodotto fisico realmente esistente. Questa metodologia permette di identificare un prodotto con un \textit{ecng} specifico.& UC1 \\
    \hline
    RFN-5.3 & Nella pagina visualizzazione prodotto deve essere possibile visualizzare informazioni riguardanti le specifiche del prodotto, i documenti al prodotto collegati ed i \textit{files} scaricabili come \gls{firmwareg}\glox o modelli 3D.& UC1 \\
    \hline
    RFN-6 & Deve essere implementata la \textit{feature} Service, in particolare \gls{servicedoveng}\glox, è richiesto che l'utente visualizzi un singolo suo prodotto identificato con un seriale& UC2 \\
    \hline
    RFN-6.1 & Nella pagina di \textit{Serviced Oven} si devono poter visualizzare le informazioni riguardanti la connessione, si deve poter avere riferimenti alla pagina Prodotto del forno, e poter visualizzare e modificare le informazioni aggiuntive legate al prodotto, in particolare informazioni legate al cliente presso cui il forno è installato.& UC2 \\
    \hline
    RFN-7 & Per tutte le piattaforme deve essere realizzato un sistema di navigazione che permetta di cambiare le sezioni della app & UC0, UC1, UC2 \\
    \hline
    RFN-7.1 & Per le piattaforme mobile è prevista la realizzazione di una \gls{tabbarg}\glox invece per la piattaforma \textit{web} è prevista la realizzazione di una \gls{navbarg}\glox & UC0, UC1, UC2 \\
    \hline
    \hiderowcolors
    \caption{Tabella del tracciamento dei requisiti funzionali.}
    \label{tab:requisiti_funzionali}
    \end{longtable}
\end{center}

\begin{center}
    \rowcolors{1}{}{tableGray}
    \begin{longtable}{|p{2.25cm}|p{7.75cm}|p{2.25cm}|}
    \hline
    %\rowcolor{hyperColor!5}
    \multicolumn{1}{|c|}{\textbf{Requisito}} & \multicolumn{1}{c|}{\textbf{Descrizione}} & \multicolumn{1}{c|}{\textbf{Use Case}}\\
    \hline 
    \endfirsthead
    \rowcolor{white}
    \multicolumn{3}{c}{{\bfseries \tablename\ \thetable{} -- Continuo della tabella}}\\
    \hline
    %\rowcolor{hyperColor!5}
    \multicolumn{1}{|c|}{\textbf{Requisito}} & \multicolumn{1}{c|}{\textbf{Descrizione}} & \multicolumn{1}{c|}{\textbf{Use Case}}\\
    \hline 
    \endhead
    \hline
    \rowcolor{white}
    \multicolumn{3}{|r|}{{Continua nella prossima pagina...}}\\
    \hline
    %\caption{Tabella del tracciamento dei requisiti qualitativi.}
    \endfoot
    \endlastfoot
    RQN-1 & La navigazione della app deve essere funzionale sia da \textit{mobile} che da \textit{web}, permettendo di indentificare con facilità il corretto flusso di funzionamento della \textit{app} indipendentemente dalla piattaforma in uso. & - \\
    \hline
    RQN-2 & Il codice prodotto deve essere scalabile e mantenibile, rispettando i \textit{design pattern} attualmente già utilizzati nel ecosistema Unox. & - \\
    \hline
    RQN-3 & Tutte le funzionalità realizzate devono essere testate su tutte le piattaforme di destinazione garantendo il loro funzionamento. & - \\
    \hline
    \hiderowcolors
    \caption{Tabella del tracciamento dei requisiti qualitativi.}
    \label{tab:requisiti_qualitativi}
    \end{longtable}
\end{center}

\begin{center}
    \rowcolors{1}{}{tableGray}
    \begin{longtable}{|p{2.25cm}|p{7.75cm}|p{2.25cm}|}
    \hline
    %\rowcolor{hyperColor!5}
    \multicolumn{1}{|c|}{\textbf{Requisito}} & \multicolumn{1}{c|}{\textbf{Descrizione}} & \multicolumn{1}{c|}{\textbf{Use Case}}\\
    \hline 
    \endfirsthead
    \rowcolor{white}
    \multicolumn{3}{c}{{\bfseries \tablename\ \thetable{} -- Continuo della tabella}}\\
    \hline
    %\rowcolor{hyperColor!5}
    \multicolumn{1}{|c|}{\textbf{Requisito}} & \multicolumn{1}{c|}{\textbf{Descrizione}} & \multicolumn{1}{c|}{\textbf{Use Case}}\\
    \hline 
    \endhead
    \hline
    \rowcolor{white}
    \multicolumn{3}{|r|}{{Continua nella prossima pagina...}}\\
    \hline
    \endfoot
    \endlastfoot
    
    RVN-1 & Le pagine e i componenti della applicazione devono essere realizzati tenendo conto delle indicazioni date dal team di \textit{Design}, rispettando un \gls{designsystemg}\glox esistente e le convenzioni date per stili grafici compresi \textit{UI/UX} & - \\
    \hline
    RVN-2 & Le \textit{feature} realizzate per la applicazione devono essere compatibili con tutte le piattaforme di destinazione. & - \\
    \hline
    \hiderowcolors
    \caption{Tabella del tracciamento dei requisiti di vincolo.}
    \label{tab:requisiti_vincolo}
    \end{longtable}
\end{center}

\newpage


