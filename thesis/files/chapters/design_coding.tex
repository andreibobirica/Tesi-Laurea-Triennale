\chapter{Progettazione e Codifica}
\label{chap:design_coding}

In questo capitolo verranno descritti i processi di progettazione e codifica utilizzati nello sviluppo dell'applicazione \gls{ddcserviceg}\glox. Si esamineranno l'architettura dell'applicazione, le tecnologie utilizzate per il frontend e il backend, i protocolli di comunicazione, l'autenticazione e l'architettura a componenti.

\section{Architettura dell'Applicazione}
\label{sec:architettura_applicazione}

\subsection{Backend e Frontend}
\label{subsec:backend_frontend}

Questa sezione descrive la suddivisione tra il backend e il frontend dell'applicazione. Include una panoramica delle tecnologie utilizzate e delle responsabilità di ciascuna parte.

\subsubsection*{Responsabilità del Backend}
Dettagliare le responsabilità del backend, come la gestione dei dati, l'autenticazione, la logica di business e le API.

\subsubsection*{Responsabilità del Frontend}
Dettagliare le responsabilità del frontend, come l'interfaccia utente, la gestione dello stato dell'applicazione e l'interazione con le API del backend.

\subsection{Struttura delle Applicazioni}
\label{subsec:struttura_applicazioni}

Descrivere la struttura complessiva delle applicazioni, includendo la disposizione delle directory e dei file, e come questi sono organizzati per mantenere il codice manutenibile e modulare.

\subsubsection*{Monorepo}
Spiegare l'uso di una monorepo per gestire il codice del progetto e come questo approccio facilita la condivisione del codice e delle risorse tra le diverse parti dell'applicazione.

\subsubsection*{Struttura delle Cartelle}
Dettagliare la struttura delle cartelle del progetto, spiegando la logica dietro la suddivisione in moduli, componenti e servizi.

\section{Comunicazione}
\label{sec:comunicazione}

\subsection{Protocolli di Comunicazione}
\label{subsec:protocolli_comunicazione}

Descrivere i protocolli di comunicazione utilizzati tra il frontend e il backend. Includere dettagli su REST, GraphQL e altri protocolli pertinenti.

\subsubsection*{REST}
Descrivere come e quando vengono utilizzate le API REST, includendo esempi di endpoint e metodi HTTP utilizzati.

\subsubsection*{GraphQL}
Descrivere l'uso di GraphQL, i vantaggi rispetto a REST, e come viene implementato nel progetto. Includere esempi di query e mutazioni.

\subsection{GraphQL Codegen}
\label{subsec:graphql_codegen}

Descrivere l'utilizzo di strumenti di code generation per GraphQL, come GraphQL Code Generator. Spiegare come questi strumenti aiutano a mantenere il codice tipizzato e sincronizzato con lo schema GraphQL.

\subsubsection*{Configurazione e Utilizzo}
Dettagliare la configurazione di GraphQL Codegen nel progetto e fornire esempi di come viene utilizzato per generare codice.

\subsection{Autenticazione}
\label{subsec:autenticazione}

Descrivere il sistema di autenticazione utilizzato nell'applicazione. Includere dettagli sui flussi di autenticazione, i token JWT e le strategie di sicurezza implementate.

\subsubsection*{Flussi di Autenticazione}
Descrivere i vari flussi di autenticazione, come l'accesso tramite username e password, l'autenticazione a due fattori, e il rinnovo dei token.

\subsubsection*{Gestione Cookie}

\subsubsection*{Gestione dei Token JWT}
Dettagliare l'uso di JSON Web Tokens (JWT) per l'autenticazione, inclusi i processi di generazione, validazione e gestione dei token.

\section{Architettura a Componenti}
\label{sec:architettura_componenti}

\subsection{Componenti di Base Design System}
Descrivere i componenti di base dell'applicazione e come vengono creati utilizzando React. Includere esempi di componenti comuni come bottoni, input e layout.

\subsection{Componenti Compositi}
Descrivere i componenti compositi, che combinano i componenti di base per formare parti più complesse dell'interfaccia utente.

\subsubsection*{Esempi di Componenti Compositi}
Fornire esempi di componenti compositi come moduli di login, tabelle di dati e dashboard.

\subsection{Gestione dello Stato}
Descrivere come viene gestito lo stato dell'applicazione utilizzando Redux o un altro sistema di gestione dello stato. Includere dettagli su azioni, riduttori e middleware.

\subsubsection*{Azioni e Riduttori}
Descrivere il ruolo delle azioni e dei riduttori nella gestione dello stato, includendo esempi pratici.

\subsubsection*{Middleware}
Descrivere l'uso dei middleware per gestire operazioni asincrone e altri effetti collaterali nello stato dell'applicazione.

\subsection{Stilizzazione dei Componenti}
Descrivere le tecniche di stilizzazione utilizzate per i componenti React, come CSS-in-JS, Styled Components, e altri approcci moderni.

\subsubsection*{Esempi di Stilizzazione}
Fornire esempi di stilizzazione dei componenti, spiegando come vengono applicati gli stili in modo modulare e riutilizzabile.

\newpage