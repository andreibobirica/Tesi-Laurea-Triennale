\chapter{Descrizione dello stage}
\label{chap:stage_descrizione}

\section{Pianificazione}
\subsection{Attività}
La seguente pianificazione delle attività è stata inizialmente delineata nel piano di lavoro. 
Tuttavia, durante lo stage, alcune attività sono state modificate sia per una maggiore comprensione emersa dall'analisi dei requisiti, sia per cambiamenti negli obiettivi da realizzare.
\subsubsection*{Prima Settimana (40 ore)}
\begin{itemize}
    \item Incontro con le persone coinvolte nel progetto per discutere i requisiti e le richieste relative al sistema da sviluppare.
    \item Verifica delle credenziali e degli strumenti di lavoro assegnati.
    \item Presa visione dell'infrastruttura esistente, in particolare della app \gls{ddc}\glox, del suo \gls{designsystemg}\glox e della \gls{monorepog}\glox esistente.
    \item Formazione sulle tecnologie adottate.
\end{itemize}

\subsubsection*{Seconda Settimana (40 ore)}
\begin{itemize}
    \item Studio del software \gls{backendg}\glox esistente con cui l'applicazione si integrerà.
    \item Avvio dello sviluppo dell'applicazione, definizione dell'architettura, dello \textit{stack} di navigazione e implementazione della funzionalità di autenticazione.
\end{itemize}

\subsubsection*{Terza Settimana (40 ore)}
\begin{itemize}
    \item Continuazione dello sviluppo dell'architettura dell'applicazione, inclusi il \textit{login} e lo stack di navigazione principale.
\end{itemize}

\subsubsection*{Quarta Settimana (40 ore)}
\begin{itemize}
    \item Sviluppo della funzionalità consultazione Prodotto, detta \textit{Product Page}
\end{itemize}

\subsubsection*{Quinta Settimana (40 ore)}
\begin{itemize}
    \item Continuazione dello sviluppo delle funzionalità di consultazione Prodotto.
\end{itemize}

\subsubsection*{Sesta Settimana (40 ore)}
\begin{itemize}
    \item Implementazione nella \textit{Product Page} delle funzionalità di visualizzazione manuali, ricambistica e \textit{Tech and Docs} 
    \item Sviluppa della funzionalità consultazione \textit{Serviced Oven} 
\end{itemize}

\subsubsection*{Settima Settimana (40 ore)}
\begin{itemize}
    \item Sviluppo della funzionalità di gestione del flusso \gls{rmag}\glox\gloxspacing.
\end{itemize}

\subsubsection*{Ottava Settimana - Conclusione (40 ore)}
\begin{itemize}
    \item Continuazione dello sviluppo della funzionalità di gestione del flusso \gls{rma}.
    \item Test e ottimizzazione dell'applicazione con il personale aziendale.
\end{itemize}

\begin{table}[ht]
    \centering
    \begin{tabular}{c|c}
    \hline\hline
    \textbf{Durata in ore} & \textbf{Descrizione dell'attività} \\
    \hline\hline
    40 & Inserimento in azienda \\
    \hline
    24 & Studio Backend esistente \\
    \hline
    56 & Sviluppo architettura applicazione\\
    \hline
    80 & Sviluppo della funzionalità \textit{Product Page} \\
    \hline
    40 & Sviluppo della funzionalità \textit{Serviced Oven Page} \\
    \hline
    60 & Sviluppo funzionalità flusso \textit{RMA} \\
    \hline
    20 & Test e ottimizzazione della applicazione \\
    \hline
    \textbf{Totale ore} & \textbf{320} \\
    \hline\hline
    \end{tabular}
    \caption{Suddivisione delle ore di lavoro per le attività di progetto.}
    \label{tab:ore-lavoro}
    \end{table}

\subsection{Obbiettivi}
\subsubsection*{Obiettivi obbligatori}
\begin{itemize}
    \item \textbf{Architettura dell'applicazione:} Definizione dello scheletro e dell'architettura dell'applicazione, compresa la navigazione.
    \item \textbf{Autenticazione: } Implementazione della funzionalità di autenticazione (\textit{SignIn, SignUp, Recover Password})
    \item \textbf{\textit{Product Page}: }Sviluppo della funzionalità consultazione Prodotto e delle funzionalità di visualizzazione manuali, ricambistica e \textit{Tech and Docs} 
    \item \textbf{\textit{Serviced Oven}: }Sviluppa della funzionalità consultazione dei propri forni in \textit{service} detti \textit{Serviced Oven}
    \item \textbf{Test piattaforme: } Esecuzione di test sulla piattaforma web e mobile per garantire la massima portabilità del codice

\end{itemize}

\subsubsection*{Obiettivi desiderabili}
\begin{itemize}
    \item \textbf{RMA:} Sviluppo della funzionalità di gestione del flusso \gls{rmag}\glox\gloxspacing.
\end{itemize}

\subsubsection*{Obiettivi facoltativi}
\begin{itemize}
    \item \textbf{Test E2E:} Creazione di test automatizzati \gls{e2eg}\glox\gloxspacing per verificare le varie componenti dell'applicazione, per massimizzare l'efficienza del processo di \textit{testing}.
\end{itemize}

\subsection{Vincoli}

Durante lo sviluppo del progetto, sono stati identificati vari vincoli che hanno influenzato il contesto operativo e le decisioni progettuali. 
Questi vincoli hanno avuto un impatto significativo sulle scelte effettuate e sull'approccio adottato per la realizzazione dell'applicazione. 
I principali vincoli sono suddivisibili in categorie come vincoli aziendali, tecnologici, temporali e di design.
\subsection*{Vincoli tecnologici}
Un vincolo importante riguardava l'adozione delle tecnologie già utilizzate da \myAzienda
L'applicazione doveva essere sviluppata utilizzando strumenti e tecnologie in uso all'interno dell'azienda per assicurare l'integrazione e la coerenza con l'ecosistema tecnologico esistente.
\subsection*{Vincoli temporali}
Un vincolo temporale significativo era la data di conclusione dello stage, fissata per il 7 giugno 2024.
Questa scadenza ha imposto un termine rigido per il completamento del progetto, richiedendo una gestione attenta del tempo e delle risorse per rispettare il limite prestabilito.
\subsection*{Vincoli di design}
I vincoli di design includevano il rispetto del \gls{designsystemg}\glox\gloxspacing aziendale e delle specifiche grafiche fornite dall'azienda.
L'applicazione doveva essere allineata al \textit{Design System} esistente e rispettare le palette di colori e le linee guida visive stabilite dall'azienda.

\section{Analisi preventiva dei rischi}
Durante la fase di analisi iniziale, sono stati individuati alcuni possibili rischi che avrebbero potuto causare problemi nel corso del progetto.
Per affrontarli, sono state elaborate delle possibili soluzioni.
\begin{risk}{Inesperienza tecnologica}
    \riskdescription{Era previsto l'utilizzo di tecnologie mai utilizzate prima, il che poteva causare rallentamenti nello sviluppo dell'applicazione}
    \risksolution{L'azienda ha programmato un periodo di circa una settimana dedicato allo studio autonomo delle tecnologie, utilizzando tutorial e risorse interne}
    \label{risk:inesperienza-tecnologica} 
\end{risk}

\begin{risk}{Difficoltà nel soddisfare le esigenze di Design}
    \riskdescription{Inizialmente era previsto realizzare le funzionalità richieste senza dare peso al design e alla parte grafica dell'app. Tuttavia, è emersa la necessità di cooperare con il team di design per seguire le loro linee guida}
    \risksolution{Si è utilizzato il \gls{designsystemg}\glox già esistente, adattandolo dove necessario, e si è dato del tempo per imparare a utilizzare nuovi strumenti come Figma}
    \label{risk:design-system} 
\end{risk}

\begin{risk}{Interpretazione dei requisiti}
    \riskdescription{I requisiti avrebbero potuto subire aggiornamenti in corso d'opera a causa della difficoltà d'individuare con facilità se un requisito fosse realizzabile o meno}
    \risksolution{Sono stati pianificati meeting regolari con i team coinvolti per discutere e individuare soluzioni, semplificando la realizzazione dei requisiti proposti}
    \label{risk:interpretazione-requisiti} 
\end{risk}

\newpage